\chapter*{Conclusion}
\addcontentsline{toc}{chapter}{Conclusion}

\marginnote{To paraphrase Alan Perlis, lispers know the value of everything and
  the cost of nothing.}

Lisps are known to be high level, functional languages. They also had a reputation for inneficiency
back in their heyday. But things have changed  and lots of ground has been covered not only for Lisp but for
functional languages in general.
\marginnote{To say the least; even back when Peter Norvig first
  wrote \textit{Artificial Intelligence: A Modern Approach}he benchmarked Lisp with favorable results}

What did not change, however, was the awe-inspiring sense of magic that using Lisp produced. In that sense,
learning how all of those nice, quote-unquote \marginnote{No pun intended} \textit{pure} trickle down
and become bits is a very enlightening experience. Likewise, the idea of incremental compiler construction
is near and dear to me, even if I didn't do it justice. To me, the notion of \textit{language} pervades
our profession, and demystifying compilers is a huge leap in the discipline of making and understanding
software and advancing our field.
