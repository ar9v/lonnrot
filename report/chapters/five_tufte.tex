\chapter{Quick Reference}\label{ch:quickref}

\section{Setup}

Make sure to have \texttt{gcc} and \texttt{nasm}. Likewise, you should have Racket installed with
\texttt{megaparsack} and \texttt{a86}:

\begin{verbatim}
raco pkg install 'https://github.com/cmsc430/www.git?path=langs#main'
raco pkg install megaparsack
\end{verbatim}

\noindent Once you have met the required dependencies, clone the repository and run \texttt{install.sh}.

\begin{verbatim}
git clone https://github.com/argvniyx/lonnrot.git
cd lonnrot && ./install.sh
\end{verbatim}

This will symlink \texttt{compass}, a CLI tool that allows to interpret (\texttt{-i}) or compile
(\texttt{-c}) Lonnrot files. A Lonnrot file need not be identified with a file extension, but the common
convention is to suffix your files with \texttt{.rot}.

\section{Core Forms and Primitives}

\subsection*{void}
0-arity primitive, simply produces the value of void. Exists as a result of effectful computation.

\subsection*{read-byte}
0-arity primitive. Reads one keystroke from the user. This uses C's \texttt{getc} function.

\subsection*{peek-byte}
Same as \texttt{read-byte} but \texttt{unget}s the char back onto \texttt{stdin}.

\subsection*{+}
\texttt{Integer x Integer -> Integer}\\
\noindent Takes two integers and produces their sum. Unlike other Schemes, this \textbf{is not n-ary}.

\subsection*{-}
\texttt{Integer x Integer -> Integer}\\
\noindent Takes two integers and produces their difference. Unlike other Schemes, this \textbf{is not n-ary}.

\subsection*{add1}
\texttt{Integer -> Integer}\\
\noindent Takes an integer $a$  and produces $a+1$.

\subsection*{sub1}
\texttt{Integer -> Integer}\\
\noindent Takes an integer $a$  and produces $a-1$.

\subsection*{zero?}
\texttt{Any -> Boolean}\\
\noindent Takes a value and returns \texttt{\#t} if it is equal to zero, \texttt{\#f} if not.

\subsection*{null?}
\texttt{Any -> Boolean}\\
\noindent Takes a value and checks if it is the empty list. If is, it returns \texttt{\#t}, \texttt{\#f}
otherwise.

\subsection*{integer?}
\texttt{Any -> Boolean}\\
\noindent Takes a value and checks if it is an integer.

\subsection*{boolean?}
\texttt{Any -> Boolean}\\
\noindent Takes a value and checks if it is a boolean.

\subsection*{char?}
\texttt{Any -> Boolean}\\
\noindent Takes a value and checks if it is a character.

\subsection*{eof-object?}
\texttt{Any -> Boolean}\\
\noindent Checks if the given value is \texttt{eof}. There is no file reading in Lonnrot yet, but we can
check this, for example, when a user types \texttt{C-c} at a prompt from \texttt{(read-byte)}.

\subsection*{string?}
\texttt{Any -> Boolean}\\
\noindent Takes a value and checks if it is a string.

\subsection*{string-length}
\texttt{String -> Integer}\\
\noindent Takes a string and returns its length.

\subsection*{string-ref}
\texttt{String x Integer -> Char}\\
\noindent Takes a string and an index and returns the character at the index, or an error if the index is out of bounds.

\subsection*{eq?}
\texttt{Any x Any -> Boolean}\\
\noindent Takes two values and checks if they are \texttt{eq?}, meaning, if they are exactly the same. This is
trivial for immediates, but not so for pointers.

\subsection*{=}
\texttt{Integer x Integer -> Boolean}\\
\noindent Same as \texttt{eq?} but for integers.

\subsection*{<}
\texttt{Integer x Integer -> Boolean}\\
\noindent Returns \texttt{\#t} if the first argument is strictly lower than the second.

\subsection*{>}
\texttt{Integer x Integer -> Boolean}\\
\noindent Returns \texttt{\#t} if the first argument is strictly greater than the second.

\subsection*{not}
\texttt{Any -> Boolean}\\
\noindent If given a truthy value, this returns \texttt{\#f}, \texttt{\#t} otherwise.

\subsection*{char->integer}
\texttt{Char -> Integer}\\
\noindent Takes an character and returns its codepoint.

\subsection*{integer->char}
\texttt{Char -> Integer}\\
\noindent Takes an integer and returns the character it represents.

\subsection*{write-byte}
\texttt{Integer -> Void}\\
\noindent Displays a single byte to \texttt{stdout}

\subsection*{displayln}
\texttt{Any -> Void}\\
\noindent Displays any data given to it to \texttt{stdout}

\subsection*{box}
\texttt{Any -> Box}\\
\noindent Boxes any value (i.e.\ creates a reference for it in memory)

\subsection*{unbox}
\texttt{Box -> Any}\\
\noindent Returns a value boxed by a Box, or produces an exception if the value given is not a box.

\subsection*{cons}
\texttt{Any x Any -> Pair}\\
\noindent Produces a pair.

\subsection*{list}
\texttt{Any\ldots -> Pair}\\
\noindent Produces a proper list. It is a shorthand for \texttt{(cons (cons (cons \ldots '())))}

\subsection*{car}
\texttt{Pair -> Any}\\
\noindent Returns the first element of a pair or produces an error if its not given a pair.

\subsection*{cdr}
\texttt{Pair -> Any}\\
\noindent Returns the second element of a pair (or rest, in the case of a list), or produces an error if not
given a pair.

\subsection*{if}
\texttt{Any x Any a x Any b -> Either a or b}\\
\noindent Evaluates its first argument and, depending on whether it is true or not, produces value a (if true)
or b (when false)

\subsection*{cond}
A sequence of clauses where the first clause with a truthful predicate returns the value of its expression.

\begin{verbatim}
(cond [(#f ''this is not returned'')]
      [(#t ''but this is'')]
     [else ''we don't even get here!''])
\end{verbatim}

\subsection*{let}
\noindent Takes a binding, which is a list of a single pair, which binds the value to the symbol given and
returns the evaluation of its body (the second argument)

\texttt{(let ((x 4)) (+ x 4)) -> 8}

\subsection*{letrec}
\noindent Like \texttt{let}, but binds a \texttt{lambda} to a symbol. See \texttt{define}.

\subsection*{define}
\noindent Syntactic sugar for \texttt{letrec}

\begin{verbatim}
;; This...
(define (double x) (+ x x))

;; ...is the same as this:
(letrec ((f (lambda (x) (+ x x)))) ...)
\end{verbatim}

\subsection*{lambda}
\noindent Produces a function pointer, which can then be applied to arguments.


\section{Standard Library}

\subsection*{length}
\texttt{Pair -> Integer}
\noindent Takes a list and returns its size.

\subsection*{append}
\texttt{Pair x Pair -> Pair}\\
\noindent Takes two lists and returns the list that contains all of the elements of both lists in the order
given.

\subsection*{map}
\texttt{Lambda x Pair -> Pair}\\
\noindent Returns the list that results of applying the first argument to each element of the list.

\subsection*{filter}
\texttt{Lambda x Pair -> Pair}\\
\noindent Returns the list of elements in the second argument that fulfill the predicate of the first argument.

\subsection*{sort}
\texttt{List Integer -> List Integer}\\
\noindent Takes a list of integers and sorts them in increasing order

\subsection*{find}
\texttt{List x Any -> Any or False}\\
\noindent Looks the second argument in the first. If it finds it, it returns the argument itself, \texttt{\#f}
otherwise.

\subsection*{and}
\texttt{Any x Any -> Boolean}\\
\noindent Takes two arguments and returns true if both are truthy (i.e.\ not false).

\subsection*{or}
\texttt{Any x Any -> Boolean}\\
\noindent Takes two arguments and returns true if either of them are truthy.
