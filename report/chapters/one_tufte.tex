\chapter{Tetragrammaton, or The Form}\label{ch:TFE}

%\newthought{This report explores} the implementation of a compiler for a miniKanren-like, logic Domain Specific Language. This includes, amongst other things, the exploration of logic programming language concepts, particularly
%those which differ from more traditional solutions like Prolog, and their implications on compiler design, drawing from Prolog's own WAM.\sidenote{WAM stands for ``Warren Abstract Machine'', in honor of its inventor, David H.
%Warren.}
%
%This chapter outlines the main components of the language in the form of \textit{tokens}, \textit{syntax diagrams} and \textit{semantics}. Then, it presents some core functions, which will serve as a foundation for a sort of
%standard library in later chapters. This discussion is finally rounded out with the consideration of \textit{data types} and \textit{data structures} that are a part of the language.

% TODO/IDEAS
% Appendix with standard lib
% Appendix with vocabulary, or maybe that could be part of the preface, abstract etc.
% Alternatively, some definitions could be part of the marginalia/sidenotes

\section{Basic Elements}\label{sec:ch1_BE}

\section{Syntax}\label{sec:ch1_Syn}

\section{Syntax Diagrams}\label{sec:ch1_SD}
