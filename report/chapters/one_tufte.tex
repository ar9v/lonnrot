\chapter{The Front End}\label{ch:one}

% Only add if there is time
% \epigraph{The s-expression syntax dates back to 1960. This syntax is often controversial amongst programmers. Observe, however, something deeply valuable that it gives us. While parsing traditional languages can be very complex,
%   parsing this syntax is virtually trivial.}{Shriram Krishnamurthi~\cite{krishnamurthi20}}

% c.2
\newthought{In this chapter} we go over the basic structural elements of Lonnrot Scheme: its lexicon and its syntax. More
importantly, we describe the way these concepts map to the actual implementation of the project, that is,
we describe how we can express these items as \textbf{parser combinators} for use with \texttt{megaparsack}.\\

% TODO: Graham Hutton quote here
\newthought{Although} it is common to express lexemes in a language as regular expressions, an interesting insight
of Chomsky's Hierarchy is the fact that given that Context Free Languages subsume Regular Languages we
can also express these as grammars (albeit simple ones).\marginnote{Indeed the idea of parser combinators, I find,
  is not that different in spirit from that of Thompson's Algorithm to convert regex to NFAs}

This might seem a gratuitous observation, but it's not: it allows for expressing atomic data as
\textbf{parser combinators}. Building a parser for a language is, consequently, building parsers for all types
of simple, more easily discernible data, and putting these together. We present thus the core lexemes
of our language as a grammar.\clearpage


\section{Lexicon}\label{sec:ch1_lex}
\setlength{\grammarparsep}{8pt} % increase separation between rules
\setlength{\grammarindent}{8em} % increase separation between LHS/RHS
\begin{grammar}
    <datum> ::= <boolean>
    \indalt <character>
    \indalt <variable>
    \indalt <string>
    \indalt <number>
    \indalt <list>

    <variable> ::= <initial> <subsequent>*

    <initial> ::= <letter> | `!' | `$' | `&' | `*' | `/' | `:' | `<' | `=' | `>' | `?' | `~' | `_' | `^'

    <letter> ::= a | \ldots | z | A | \ldots | Z

    <boolean> ::= \texttt{\#t} | \texttt{\#f}

    <number> ::= <integer> | <decimal>

    <integer> ::= <digit>+

    <decimal> ::= <integer> `.' <digit>+

    <char> ::= `\#' <any-character>

    <list> ::= (<datum>*)
    \indalt (<datum>+ . <datum>)
\end{grammar}

\clearpage

% c.3
% Make marginnote that talks about how formals do not include pairs
% Note that application does not take an arbitrary expression, but a variable
% Explain that variables includes other primitives, such as +, -, not, eq? etc.
\section{Syntax}\label{sec:ch1_syn}
\setlength{\grammarparsep}{8pt} % increase separation between rules
\setlength{\grammarindent}{8em} % increase separation between LHS/RHS
\begin{grammar}
    <program>     ::= <form>*

    <form>        ::= <definition> | <expression>

    <definition>  ::= (\texttt{define} <variable> <expression>)

    <expression>  ::= <constant>
    \indalt <variable>
    \indalt (\texttt{quote} <datum>)
    \indalt (\texttt{fresh} <formals> <expression> <expression>*)
    \indalt (\texttt{==} <expression> <expression>)
    \indalt (\texttt{run} <digit> <formals> <expression>)
    \indalt (\texttt{lambda} <formals> <expression> <expression>*)
    \indalt (\texttt{if} <expression> <expression> <expression>)
    \indalt (\texttt{set!} <variable> <expression>)
    \indalt <application>

    <constant>    ::= <boolean> | <number> | <char> | <string>

    <formals>     ::= <variable>
    \indalt (<variable>*)
    \indalt (<variable> <variable>* . <variable>)

    <application> ::= (<expression> <expression>*)
\end{grammar}

\clearpage

\section{Syntax Diagrams}\label{sec:ch1_SD}
\subsection{Tokens}
\railalias{dol}{\$}
\railalias{amp}{\&}
\railalias{til}{\~{}}
\railalias{uscore}{\textunderscore}
\railalias{caret}{\^{}}
\railalias{bs}{\char''5C}
\railterm{dol, amp, til, uscore, caret, bs}
\begin{rail}
  datum : boolean
  | character
  | variable
  | string
  | number
  | list
  ;

  boolean : '\#t' | '\#f' ;

  char : anyCharacter ;

  number : integer | decimal ;

  integer : digit + ;

  decimal : integer '.' digit + ;

  variable : initial (subsequent*) ;

  initial : letter | '!' | dol | amp | '*' | '/' | ':' | '<' | '=' | '>' | '?' | til | uscore | caret ;

  letter : [a--z] | [A--Z] ;


  list : '(' (datum *) ')' | '(' (datum +) '.' datum ')'

\end{rail}


\subsection{Grammar}
\railalias{dol}{\$}
\railalias{amp}{\&}
\railalias{til}{\~{}}
\railalias{uscore}{\textunderscore}
\railalias{caret}{\^{}}
\railalias{bs}{\char''5C}
\railterm{dol, amp, til, uscore, caret, bs}
\begin{rail}
  program : form* ;

  form : definition | expression ;

  definition : '(' 'define' variable expression ')' ;

  expression : constant
  | variable
  | '(' 'quote' datum ')'
  | '(' 'fresh' formals expression (expression*) ')'
  | '(' '==' expression expression ')'
  | '(' 'run' digit formals expression ')'
  | '(' 'lambda' formals expression (expression*) ')'
  | '(' 'if' expression expression expression ')'
  | '(' 'set!' variable expression ')'
  | application ;

  formals : variable
  | '(' (variable*) ')'
  | '(' variable (variable*) '.' variable ')' ;

  application : '(' expression (expression*) ')' ;

\end{rail}

