\chapter{Tetragrammaton}\label{ch:TFE}

\epigraph{The s-expression syntax dates back to 1960. This syntax is often controversial amongst programmers. Observe, however, something deeply valuable that it gives us. While parsing traditional languages can be very complex,
  parsing this syntax is virtually trivial.}{Shriram Krishnamurthi~\cite{krishnamurthi20}}

%This chapter outlines the main components of the language in the form of \textit{tokens}, \textit{syntax diagrams} and \textit{semantics}. Then, it presents some core functions, which will serve as a foundation for a sort of
%standard library in later chapters. This discussion is finally rounded out with the consideration of \textit{data types} and \textit{data structures} that are a part of the language.


% Mention we won't deal with strings for now
% Mention anyChar is literally any char, digit is 0 to 9

\section{Basic Elements}\label{sec:ch1_BE}
\setlength{\grammarparsep}{8pt} % increase separation between rules
\setlength{\grammarindent}{8em} % increase separation between LHS/RHS
\begin{grammar}
    <datum> ::= <boolean>
    \indalt <character>
    \indalt <variable>
    \indalt <string>
    \indalt <number>
    \indalt <list>

    <variable> ::= <initial> <subsequent>*

    <initial> ::= <letter> | `!' | `$' | `&' | `*' | `/' | `:' | `<' | `=' | `>' | `?' | `~' | `_' | `^'

    <letter> ::= a | \ldots | z | A | \ldots | Z

    <boolean> ::= \texttt{\#t} | \texttt{\#f}

    <number> ::= <integer> | <decimal>

    <integer> ::= <digit>+

    <decimal> ::= <integer> `.' <digit>+

    <char> ::= `\#' <any-character>

    <list> ::= (<datum>*)
    \indalt (<datum>+ . <datum>)
\end{grammar}


\section{Syntax}\label{sec:ch1_Syn}
\setlength{\grammarparsep}{8pt} % increase separation between rules
\setlength{\grammarindent}{8em} % increase separation between LHS/RHS
\begin{grammar}
    <program>     ::= <form>*

    <form>        ::= <definition> | <expression>

    <definition>  ::= (\texttt{define} <variable> <expression>)

    <expression>  ::= <constant>
    \indalt <variable>
    \indalt (\texttt{quote} <datum>)
    \indalt (\texttt{fresh} <formals> <expression> <expression>*)
    \indalt (\texttt{==} <expression> <expression>)
    \indalt (\texttt{run} <digit> <formals> <expression>)
    \indalt (\texttt{lambda} <formals> <expression> <expression>*)
    \indalt (\texttt{if} <expression> <expression> <expression>)
    \indalt (\texttt{set!} <variable> <expression>)
    \indalt <application>

    <constant>    ::= <boolean> | <number> | <char> | <string>

    <formals>     ::= <variable>
    \indalt (<variable>*)
    \indalt (<variable> <variable>* . <variable>)

    <application> ::= (<expression> <expression>*)
\end{grammar}

\clearpage

\section{Syntax Diagrams}\label{sec:ch1_SD}
\railalias{dol}{\$}
\railalias{amp}{\&}
\railalias{til}{\~{}}
\railalias{uscore}{\textunderscore}
\railalias{caret}{\^{}}
\railalias{bs}{\char''5C}
\railterm{dol, amp, til, uscore, caret, bs}
\begin{rail}
  datum : boolean
  | character
  | variable
  | string
  | number
  | list
  ;

  boolean : '\#t' | '\#f' ;

  char : anyCharacter ;

  number : integer | decimal ;

  integer : digit + ;

  decimal : integer '.' digit + ;

  variable : initial (subsequent*) ;

  initial : letter | '!' | dol | amp | '*' | '/' | ':' | '<' | '=' | '>' | '?' | til | uscore | caret ;

  letter : [a--z] | [A--Z] ;


  list : '(' (datum *) ')' | '(' (datum +) '.' datum ')'

\end{rail}

\railalias{dol}{\$}
\railalias{amp}{\&}
\railalias{til}{\~{}}
\railalias{uscore}{\textunderscore}
\railalias{caret}{\^{}}
\railalias{bs}{\char''5C}
\railterm{dol, amp, til, uscore, caret, bs}
\begin{rail}
  program : form* ;

  form : definition | expression ;

  definition : '(' 'define' variable expression ')' ;

  expression : constant
  | variable
  | '(' 'quote' datum ')'
  | '(' 'fresh' formals expression (expression*) ')'
  | '(' '==' expression expression ')'
  | '(' 'run' digit formals expression ')'
  | '(' 'lambda' formals expression (expression*) ')'
  | '(' 'if' expression expression expression ')'
  | '(' 'set!' variable expression ')'
  | application ;

  formals : variable
  | '(' (variable*) ')'
  | '(' variable (variable*) '.' variable ')' ;

  application : '(' expression (expression*) ')' ;

\end{rail}

