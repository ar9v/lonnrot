\usepackage{amsmath}
\usepackage{amssymb}
\usepackage{cancel}
\usepackage{amsthm}
\usepackage{mathrsfs}
\usepackage[dvipsnames]{xcolor}
\usepackage{graphicx}
\usepackage{epigraph}
\usepackage{booktabs}
\usepackage{relsize}
\usepackage{hyperref}
\usepackage{multicol}
  \setlength{\columnsep}{3cm}
\usepackage{floatrow}
	\newfloatcommand{capbtabbox}{table}[][\FBwidth]
\usepackage{kbordermatrix}
\usepackage{transparent}
\usepackage[normalem]{ulem}
\hypersetup{colorlinks} 
\hypersetup{ %uncomment to override tufte colors
    colorlinks=true,    
    %urlcolor=ProcessBlue,
    linkcolor =RubineRed,
    citecolor=RubineRed
}

\usepackage{array}

\usepackage[shortlabels]{enumitem}
\setlist[enumerate]{leftmargin=*, align=left}
\setlist[itemize]{topsep=1ex}

% Diagram Packages: Need to keep in this order.
\usepackage{tikz}
\usetikzlibrary{cd,calc, arrows, shapes, matrix, positioning, intersections, decorations.markings, arrows.meta, decorations.pathmorphing}
  \tikzcdset{arrow style=tikz, diagrams={>=stealth}}
	\tikzstyle{vertex}=[fill=black,circle,scale=0.6]
	\tikzset{thick/.style={line width=.6mm}}
  \tikzstyle{tensor}=[circle,thick,draw=black,fill=blue!60!green!40!white,minimum size=4mm]
  \tikzstyle{witharrow} = [thick,decoration={
      markings,mark=at position 0.75 with {\arrow[scale=1.5,>=stealth]{>}}},postaction={decorate}]
  \tikzstyle{littletensor}=[circle,thick,draw=black,fill=red!30,minimum size=.5mm] 

% ----------------------------------Commands---------------------------------- %

% functional version of the old command $\iso$. This takes an argument for what goes over the arrow.
\newcommand{\ar}[1]{\xrightarrow{\ensuremath{#1}}}


% Shorthand
\newcommand{\V}{\mathcal{V}}
% \renewcommand{\C}{\mathsf{C}}
\newcommand{\D}{\mathsf{D}}
\newcommand{\E}{\mathsf{E}}
\renewcommand{\o}{{\color{YellowOrange}o}}
\newcommand{\g}{{\color{Green}g}}
\newcommand{\p}{{\color{Purple}p}}
\newcommand{\<}{\langle}
\renewcommand{\>}{\rangle}
\newenvironment{bsmallmatrix}
  {\left[\begin{smallmatrix}}
  {\end{smallmatrix}\right]}
\let\emph\relax % there's no \RedeclareTextFontCommand
\DeclareTextFontCommand{\emph}{\bfseries}




% Categories and Operators
\newcommand{\End}{\operatorname{End}}
\newcommand{\op}{\operatorname{op}}
\newcommand{\tr}{\operatorname{tr}}
\DeclareMathOperator{\id}{id}
\DeclareMathOperator{\eval}{eval}

\newcommand{\adj}[4]{
\begin{tikzcd}[ampersand replacement=\&, column sep=4ex, cramped]
   #1 \colon #2	\ar[yshift=+.6ex]{r}
\& #3 \colon #4	\ar[yshift=-.4ex]{l}
\end{tikzcd}
}

% -----------------------------------Style------------------------------------ %
% --                                                                        -- %

\theoremstyle{plain}
\newtheorem{theorem}{Theorem}%[chapter]
\newtheorem{lemma}{Lemma}[chapter]
\newtheorem{proposition}{Proposition}[chapter]
\newtheorem{corollary}{Corollary}[chapter]
\newtheorem{takeaway}{Takeaway}%[section]


\theoremstyle{definition}
\newtheorem{definition}{Definition}[chapter]
\newtheorem{example}{Example}[chapter]
\newtheorem{remark}{Remark}[chapter]
\newtheorem{aside}{Aside}%[chapter]

% Named Theorems
\newtheorem{Yoneda}{Yoneda Lemma}[theorem]
\newtheorem{tych}{Tychonoff's Theorem}{\bfseries}{\itshape}
\newtheorem*{UP_vect}{Universal Property for Vector Spaces}{\bfseries}{\itshape}
\newtheorem*{UP_meet}{Universal Property for Free Meet Semilattices}{\bfseries}{\itshape}
\newtheorem*{UP_join}{Universal Property for Free Join Semilattices}{\bfseries}{\itshape}
\newtheorem*{UP_cocomplete}{Universal Property for the Free Cocompletion of a Category}{\bfseries}{\itshape}
\newtheorem*{UP_complete}{Universal Property for the Free Completion of a Category}{\bfseries}{\itshape}
\newtheorem*{mainprocedure}{Main Procedure}{\bfseries}{\itshape}

% -----------------------------------Colors----------------------------------- %

\newcommand{\tai}[1]{{\color{cyan}#1}}
\newcommand{\green}[1]{{\color{Green}#1}}

% -------------------------------TUFTE INDEX--------------------------------- %

