\chapter*{About the Lonnrot Language}
\addcontentsline{toc}{chapter}{About the Lonnrot Language}

% Discuss objective
\newthought{The Lonnrot Language} is an implementation of a compiler for a
subset of Scheme extended with the core of miniKanren. As such, it belongs to the
functional and relational paradigms of programming. This project is intended to be,
above all things, \textit{exploratory}. That implies several things, chief amongst which
are that it is not the be-all end-all of implementations, nor is it a breakthrough or anything like that.
It is quite simply a first approximation to compiler building (for me) and in that sense
it synthesizes my learning from resources I've found and have yet to find along the way.

That being said, there \textit{is} some sort of value proposition to this project. Mainly, most
if not all implementations of miniKanren to date are interpreters or are DSL's for their
respective host languages. This is understandable, especially when it comes to Lisps:
extending the Lisp language via macros is a well documented tradition, and the resulting
product is undeniably beautiful. However, there are things to be gained and interesting
insights to be had: is it worth compiling miniKanren? Can we transfer our knowledge of
methods used for other languages to this project? Can this shed light on how to optimize
some of the workings that deal with the relational semantics?\marginnote{For example,
  Will Byrd has talked about thread safety and the potential of parallelization in
  miniKanren~\cite{byrd15}}

I do not claim to have answers to these questions, and frankly I know better than to expect to have
them four months from now. What I do know is that it is an interesting premise for a
first incursion in this domain (again, for me): \textbf{if not know, when would I start?} In the
same vein, I hope that this report embodies the philosophy of growing a body of knowledge, of
allowing oneself to err and to be playful. Speaking of which \ldots

% Discuss area
\newthought{This report is divided} into four chapters which cover the steps involved in writing
the compiler. This organization is merely for editing purposes: the construction of Lonnrot is
done following the ideas presented by Abdulaziz Ghuloum in \textit{An Incremental Approach to Compiler
  Construction}.\marginnote{Lindsey Kuper \cite{kuper19} talks about this approach and Kent Dybvig's
  ``cousin'' approach of writing the compiler back-to-front, making a language progressively more
high-level, for those interested.}
The main idea is to write multiple compilers, with each one covering a progressively larger language.
This is opposed to the idea of writing one compiler, front to back, for the originally intended
language. The details and implications of this approach are very interesting and I would
do them a disservice by trying to explain them in a paragraph; I would encourage anyone
to read up on them. For now it's enough to say that this project is a sort of mishmash of ideas
on compiler construction: from Ghuloum's approach, to what makes the WAM be the WAM, passing
through what makes miniKanren different from Prolog.

In any case, that means that each chapter covers the progression of each part of the compiler.
For example, Chapter~\ref{ch:TFE} explores the lexicon and the grammar of the language, which
at first will be restrained to numbers, maybe then symbols, and so on. Chapter~\ref{ch:RSM} talks
about the semantics of the language, so the first implementations will omit the relational aspect,
eventually including it. The same idea applies for Chapters~\ref{ch:RdTIR} and~\ref{ch:TLVR}, which
talk about intermediate representations and the target languages respectively.

\newthought{Lastly, the last part of this report} comprises conclusions and some interesting
miscellanea. The conclusion will naturally discuss the state of affairs by the time the course is over.
The Appendix will have tidbits of information that might not fit the natural flow of the report.
As of the time of this writing I'm not entirely sure what I want to include there, but
stuff like the language's namesake will be explained there, for those who may be curious.
