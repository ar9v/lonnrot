\chapter*{About the Lonnrot Language}
\addcontentsline{toc}{chapter}{About the Lonnrot Language}

% Language Description
% b.1, b.2
\section{Language Overview}
\newthought{Lonnrot Scheme} is an implementation of a subset of Scheme written in Racket.
As such, it can be described as a \textbf{strict}, \textbf{call-by-value} language with
\textbf{dynamic typing}. It has \textit{atomic} data such as \textbf{integers}, \textbf{chars} and
\textbf{booleans}, \textit{structured} data such as \textbf{strings}, \textbf{boxes} and
\textbf{cons cells} and, last but not least, it has \textit{first class functions} in the form
of \textbf{lambdas}. Core forms such as \texttt{let} and \texttt{if}, as well as arithmetic primitives
such as addition and substraction are provided to manipulate the aforementioned data types, and
are discussed at greater lengths in the rest of the report.\\

% b.3
\section{Errors in Lonnrot}
\newthought{Given Lonnrot Scheme's nature} as a dynamic language and given that it targets
x86, errors are mostly \textbf{runtime exceptions}.\marginnote{There is an interesting
  exception with regards to strings: since as of the time of this writing the only way to generate
  a string is through a string literal, we can get away with a bit of static analysis when using
  \texttt{string-ref}: we can compute the length of the string at hand and as such an out-of-bounds error
is technically static (which is, admittedly, a hack).}

Type checking specifics are given in subsequent
chapters, but suffice it to say that you may stumble upon one of these types of errors when
using Lonnrot (broadly and obviously speaking):

\begin{itemize}
  \item \textbf{Syntactic:} Some core forms expect a distinct sequence of lists. For example,
        \texttt{define} expects to have a list of symbols as its second element;
        \texttt{cond} expects a finalizing \texttt{else} clause, \texttt{letrec} expects its
        second element to be a list of 2-lists (i.e.\ the bindings).
  \item \textbf{Semantic:} Primitives such as addition and substraction expect integers, whereas
        others may require characters. Conditional forms (i.e. \texttt{if} and \texttt{cond}) are
        usually not a problem in this regard, everything that is not \texttt{\#f} is considered
        \textbf{true}.
  \item \textbf{Memory:} Lonnrot does not implement garbage collection, so it is completely possible
        to have a segfault. That being said, Lonnrot does implement \texttt{tail-call optimization},
        which greatly saves stack space. However, for example, if you had an enormous list, you'd eventually
        run out of memory.
\end{itemize}

