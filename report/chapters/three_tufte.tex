\chapter{Execution}\label{ch:exe}

\newthought{Once we have translated code} what is left is evaluating it. In this
chapter, we discuss the implications of evaluating Lonnrot code: type checking,
runtime operations and how what concessions have to be made to represent an environment
in a setting such as x86 or a register-based virtual machine.

% c.4
\section{Semantics}\label{sec:ch3_semantics}
\subsection{Type Tags}
To be able to perform type-checking we must have a mechanism to distinguish values
at runtime. To that end, we use \textbf{type tags}. Under that premise, values are either
\textbf{immediates} or \textbf{pointers}.\marginnote{Immediates are called that way because they
  represent data that is storable in a machine word.}
The following table outlines the hierarchy and the corresponding tags used

\begin{center}
  \begin{tabular}[h]{c | c}
    \toprule
    \multicolumn{2}{c}{Immediates (end in 000)}\\
    \textbf{Type} & \textbf{Tag}\\
    \midrule
    Integer & 0000 0000 \\
    Char    & 0000 1000 \\
    True    & 0001 1000 \\
    False   & 0011 1000 \\
    EOF     & 0101 1000 \\
    Void    & 0111 1000 \\
    Empty   & 1001 1000 \\
    \bottomrule
  \end{tabular}
\end{center}

\begin{center}
  \begin{tabular}[h]{c | c}
    \toprule
    \multicolumn{2}{c}{Pointers (do not end in 000)}\\
    \textbf{Type} & \textbf{Tag}\\
    \midrule
    Box         & 001 \\
    Pair (Cons) & 010 \\
    Strings     & 011 \\
    Functions   & 100 \\
    \bottomrule
  \end{tabular}
\end{center}
\vspace{1cm}

To type check, then, is to emit code that:
\begin{itemize}
  \item Moves a value to a scratch register (\texttt{r9})
  \item \texttt{and}s this with a type mask (e.g.\ the type mask for an int is \texttt{1111})
  \item Compares the result with the type tag
  \item Generates a \texttt{jne} code that jumps to the runtime exception handler.
\end{itemize}
\vspace{1cm}

Type tagging raises the issue of compiler correctness due to limitations in representation.
\marginnote{Actually, the mere fact of storing integers within a machine word without any
implementation of bignums also does this, but alas.}
In Lonnrot Scheme \textit{there are integer overflows and underflows}, and the behavior is
the same as that of, say, C. Likewise, it speaks to some of the characteristics of the x86
architecture. The most obvious example of this is pointer tagging: choosing to represent
pointers as values with the last three bits set to things other than zero is not arbitrary.
This is because x86 is \textit{byte-addressable}, meaning that it is not possible for us to
accidentally ``step on'' another address by fiddling with the last three bits. We have thus
the ability to represent 8 different types of data with pointers.

In any case, further information about which primitives accept which type of data is given
in the Quick Reference~\ref{ch:quickref}.

\subsection{Function Calling}
Another important part of discussing the language's semantics is the issue of \textit{how} functions
are evaluated. Generally, there are three main issues at hand: (1) do we evaluate \textit{all} function
arguments before jumping to the function's code? (2) Which argument do we evaluate first? Should we
do this left to right? (3) Do we check if the function exists before or after evaluating the arguments?

To cut straight to the chase, the answer to those questions is: Lonnrot Scheme evaluates all of the
arguments before the jump, and it does so before checking if the function exists.
\marginnote{This is called, quite appropriately, \textbf{eval-apply} semantics of function application}
Regarding question number two, there is no particularly defined way to do this in Scheme, generally. However
Lonnrot Scheme evaluates arguments \textbf{left to right}.

% d.2
\section{Runtime and Memory}
The runtime allows us to perform certain operations that would otherwise require lots of assembly code.
The prime example of this is printing results, where we piggy-back on \texttt{printf} and a couple of
helper functions that allow for special formatting depending on the type of data to be represented. This
is especially handy because of type tagging: we can delegate the bit shifting required to produce an
actual value to the runtime instead of emitting more code ourselves (even if it is possible to do so).

Equally important though is the fact that  the runtime allows us to allocate heap memory through the use of
\texttt{malloc}.
This should make it apparent why the discussion on representing memory during compilation was
so short: because we are already thinking in terms of the actual hardware from the beginning! When we generate
code, we have to think about how we alter \texttt{rsp} and \texttt{rbx} (our heap pointer).
So, in lieu of reproducing what can be found in tons of x86 manuals, a brief description of the core
elements of the x86 architecture that come into play in our compiler are listed below:

\begin{center}
  \begin{tabular}[h]{c | c}
    \toprule
    \textbf{Register} & \textbf{Role}\\
    \midrule
    rax & Stores the result of intermediate and final computations \\
    rbx & Stores the heap pointer\\
    rcx & Scratch register, used to store the size of a closure\\
    rdx & Used to pass the heap pointer to the runtime before finishing\\
    r8  & Scratch register (used when compiling binary primitives)\\
    r9  & Scratch register (used for intermediate values in assertions)\\
    rsp & Stores the stack pointer \\
    rdi & By SysV ABI convention, this is the first argument register \\
    \bottomrule
  \end{tabular}
\end{center}
